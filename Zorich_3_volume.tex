\documentclass[12pt,a4paper,fleqn]{article}
\usepackage[utf8]{inputenc}
\usepackage[russian]{babel}
\usepackage{amssymb, amsmath, multicol}
\usepackage{enumitem}
\usepackage{lipsum}
\usepackage{euler}
\oddsidemargin=-15.4mm
\textwidth=190mm
\headheight=-32.4mm
\textheight=277mm
\parindent=0pt
\parskip=8pt
\pagestyle{empty}
\begin{document}
$y = $$\frac{x}{sin((x \cdot (x + 1)))}$

 Продиффиринцируем эту функцию

Оказывается\\
($x$)` = $1$\\
Продвинутый читатель уже заметил, что\\
($x$)` = $1$\\
Поэтому\\
($x$)` = $1$\\
Очевидно, что\\
($1$)` = $0$\\
При этом\\
($(x + 1)$)` = $(1 + 0)$\\
Говорят\\
($(x \cdot (x + 1))$)` = $((1 \cdot (x + 1)) + (x \cdot (1 + 0)))$\\
С другой стороны\\
($sin((x \cdot (x + 1)))$)` = $(((1 \cdot (x + 1)) + (x \cdot (1 + 0))) \cdot cos((x \cdot (x + 1))))$\\
С другой стороны\\
($\frac{x}{sin((x \cdot (x + 1)))}$)` = $\frac{((1 \cdot sin((x \cdot (x + 1)))) - ((((1 \cdot (x + 1)) + (x \cdot (1 + 0))) \cdot cos((x \cdot (x + 1)))) \cdot x))}{(sin((x \cdot (x + 1))) \cdot sin((x \cdot (x + 1))))}$\\


Упростим получившееся выражение

Поэтому\\
($(1 + 0)$) = $1$\\


 Получаем выражение

$\frac{((1 \cdot sin((x \cdot (x + 1)))) - ((((1 \cdot (x + 1)) + (x \cdot 1)) \cdot cos((x \cdot (x + 1)))) \cdot x))}{(sin((x \cdot (x + 1))) \cdot sin((x \cdot (x + 1))))}$
\end{document}