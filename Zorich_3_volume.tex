\documentclass[12pt,a4paper,fleqn]{article}
\usepackage[utf8]{inputenc}
\usepackage[russian]{babel}
\usepackage{amssymb, amsmath, multicol}
\usepackage{enumitem}
\usepackage{lipsum}
\usepackage{euler}
\oddsidemargin=-15.4mm
\textwidth=190mm
\headheight=-32.4mm
\textheight=277mm
\parindent=0pt
\parskip=8pt
\pagestyle{empty}
\begin{document}
Поэтому\\
($x$)` = $1$\\
Продвинутый читатель уже заметил, что\\
($x$)` = $1$\\
Поэтому\\
($x$)` = $1$\\
Отметим, что\\
($(x \cdot x)$)` = $((1 \cdot x) + (x \cdot 1))$\\
Оказывается\\
($sin((x \cdot x))$)` = $(((1 \cdot x) + (x \cdot 1)) \cdot cos((x \cdot x)))$\\
Очевидно, что\\
($(x \cdot sin((x \cdot x)))$)` = $((1 \cdot sin((x \cdot x))) + (x \cdot (((1 \cdot x) + (x \cdot 1)) \cdot cos((x \cdot x)))))$\\

\end{document}