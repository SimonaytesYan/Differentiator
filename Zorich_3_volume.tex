\documentclass[12pt,a4paper,fleqn]{article}
\usepackage[utf8]{inputenc}
\usepackage[russian]{babel}
\usepackage{amssymb, amsmath, multicol}
\usepackage{enumitem}
\usepackage{lipsum}
\usepackage{euler}
\oddsidemargin=-15.4mm
\textwidth=190mm
\headheight=-32.4mm
\textheight=277mm
\parindent=0pt
\parskip=8pt
\pagestyle{empty}
\begin{document}
$y = $$\frac{sin(ln(x^{cos(x)}))}{ln((1 - (5 \cdot x^{3})))}$\newpage \textbf{\Huge Продиффиринцируем эту функцию}

Имеем

\begin{center}
($x$)`
 = $1$\end{center}
И хотя клуб любителей таких формул двумя блоками ниже, мы продолжаем

\begin{center}
($x$)`
 = $1$\end{center}
\\ title{не сложно заметить} 

\begin{center}
($cos(x)$)`
 = $(1 \cdot (0 - sin(x)))$\end{center}
Поэтому

\begin{center}
($x^{cos(x)}$)`
 = $(x^{cos(x)} \cdot ((\frac{cos(x)}{x} \cdot 1) + ((1 \cdot (0 - sin(x))) \cdot ln(x))))$\end{center}
Отметим, что

\begin{center}
($ln(x^{cos(x)})$)`
 = $(\frac{1}{x^{cos(x)}} \cdot (x^{cos(x)} \cdot ((\frac{cos(x)}{x} \cdot 1) + ((1 \cdot (0 - sin(x))) \cdot ln(x)))))$\end{center}
Откуда

\begin{center}
($sin(ln(x^{cos(x)}))$)`
 = $((\frac{1}{x^{cos(x)}} \cdot (x^{cos(x)} \cdot ((\frac{cos(x)}{x} \cdot 1) + ((1 \cdot (0 - sin(x))) \cdot ln(x))))) \cdot cos(ln(x^{cos(x)})))$\end{center}
По теореме Эскобара

\begin{center}
($1$)`
 = $0$\end{center}
По теореме Эскобара

\begin{center}
($5$)`
 = $0$\end{center}
Телец в козероге, поэтому

\begin{center}
($x$)`
 = $1$\end{center}
[Данные удалены]

\begin{center}
($3$)`
 = $0$\end{center}
Если посмотреть на выражение под другим углом, можно получить

\begin{center}
($x^{3}$)`
 = $(x^{3} \cdot ((\frac{3}{x} \cdot 1) + (0 \cdot ln(x))))$\end{center}
Вы не шокированы?

\begin{center}
($(5 \cdot x^{3})$)`
 = $((0 \cdot x^{3}) + (5 \cdot (x^{3} \cdot ((\frac{3}{x} \cdot 1) + (0 \cdot ln(x))))))$\end{center}
Оказывается

\begin{center}
($(1 - (5 \cdot x^{3}))$)`
 = $(0 - ((0 \cdot x^{3}) + (5 \cdot (x^{3} \cdot ((\frac{3}{x} \cdot 1) + (0 \cdot ln(x)))))))$\end{center}
Если вы не понимаете этот переход, то я вам сочувствую

\begin{center}
($ln((1 - (5 \cdot x^{3})))$)`
 = $(\frac{1}{(1 - (5 \cdot x^{3}))} \cdot (0 - ((0 \cdot x^{3}) + (5 \cdot (x^{3} \cdot ((\frac{3}{x} \cdot 1) + (0 \cdot ln(x))))))))$\end{center}
Поэтому

\begin{center}
($\frac{sin(ln(x^{cos(x)}))}{ln((1 - (5 \cdot x^{3})))}$)`
 = $\frac{((((\frac{1}{x^{cos(x)}} \cdot (x^{cos(x)} \cdot ((\frac{cos(x)}{x} \cdot 1) + ((1 \cdot (0 - sin(x))) \cdot ln(x))))) \cdot cos(ln(x^{cos(x)}))) \cdot ln((1 - (5 \cdot x^{3})))) - ((\frac{1}{(1 - (5 \cdot x^{3}))} \cdot (0 - ((0 \cdot x^{3}) + (5 \cdot (x^{3} \cdot ((\frac{3}{x} \cdot 1) + (0 \cdot ln(x)))))))) \cdot sin(ln(x^{cos(x)}))))}{(ln((1 - (5 \cdot x^{3}))) \cdot ln((1 - (5 \cdot x^{3}))))}$\end{center}
\newpage \textbf{\LARGE Упростим получившееся выражение}

\begin{center}
$(\frac{cos(x)}{x} \cdot 1)$=$\frac{cos(x)}{x}$\end{center}
\begin{center}
$(1 \cdot (0 - sin(x)))$=$(0 - sin(x))$\end{center}
\begin{center}
$(0 \cdot x^{3})$=$0$\end{center}
\begin{center}
$(\frac{3}{x} \cdot 1)$=$\frac{3}{x}$\end{center}
\begin{center}
$(0 \cdot ln(x))$=$0$\end{center}
\begin{center}
$(\frac{3}{x} + 0)$=$\frac{3}{x}$\end{center}
\begin{center}
$(0 + (5 \cdot (x^{3} \cdot \frac{3}{x})))$=$(5 \cdot (x^{3} \cdot \frac{3}{x}))$\end{center}
\newpage \textbf{\Huge Получаем выражение}

$\frac{((((\frac{1}{x^{cos(x)}} \cdot (x^{cos(x)} \cdot (\frac{cos(x)}{x} + ((0 - sin(x)) \cdot ln(x))))) \cdot cos(ln(x^{cos(x)}))) \cdot ln((1 - (5 \cdot x^{3})))) - ((\frac{1}{(1 - (5 \cdot x^{3}))} \cdot (0 - (5 \cdot (x^{3} \cdot \frac{3}{x})))) \cdot sin(ln(x^{cos(x)}))))}{(ln((1 - (5 \cdot x^{3}))) \cdot ln((1 - (5 \cdot x^{3}))))}$
\end{document}