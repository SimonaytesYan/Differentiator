\documentclass[12pt,a4paper,fleqn]{article}
\usepackage[utf8]{inputenc}
\usepackage[russian]{babel}
\usepackage{amssymb, amsmath, multicol}
\usepackage{enumitem}
\usepackage{lipsum}
\usepackage{euler}
\oddsidemargin=-15.4mm
\textwidth=190mm
\headheight=-32.4mm
\textheight=277mm
\parindent=0pt
\parskip=8pt
\pagestyle{empty}
\begin{document}
\begin{center}
\textbf{\LARGE Исследовательская работа по теме:

Исследование функции дифференциальными методами}\end{center}\newpage\textbf{\LARGE Глава I. Функция}

\begin{center}
$y = $$\frac{sin(ln(x^{cos(x)}))}{ln((1 - (5 \cdot x^{3})))}$\end{center}
\newpage \textbf{\LARGE Глава II. Зрительный анализ функции}

Откуда

\begin{center}
$y = $$\frac{sin(ln(x^{cos(x)}))}{ln((1 - (5 \cdot x^{3})))}$\end{center}
\newpage \textbf{\LAGRE Глава III. Дифференцирование}

Хорошо там, где производной нет

\begin{center}
 ($x)`
 =  = 1$\end{center}
Господи, да для кого я вообще стараюсь

\begin{center}
 ($x)`
 =  = 1$\end{center}
Поэтому

\begin{center}
 ($cos(x))`
 =  = (1 \cdot (0 - sin(x)))$\end{center}
Как будет доказано в следующем семестре

\begin{center}
A = $((\frac{cos(x)}{x} \cdot 1) + ((1 \cdot (0 - sin(x))) \cdot ln(x)))$\end{center}
\begin{center}
 ($x^{cos(x)})`
 =  = (x^{cos(x)} \cdot A)$\end{center}
В ближайшее время ожидаются осадки из ваших слёз от попыток понять этот переход

\begin{center}
A = $((\frac{cos(x)}{x} \cdot 1) + ((1 \cdot (0 - sin(x))) \cdot ln(x)))$\end{center}
\begin{center}
 ($ln(x^{cos(x)}))`
 =  = (\frac{1}{x^{cos(x)}} \cdot (x^{cos(x)} \cdot A))$\end{center}
[Данные удалены]

\begin{center}
A = $((\frac{cos(x)}{x} \cdot 1) + ((1 \cdot (0 - sin(x))) \cdot ln(x)))$\end{center}
\begin{center}
B = $(\frac{1}{x^{cos(x)}} \cdot (x^{cos(x)} \cdot A))$\end{center}
\begin{center}
 ($sin(ln(x^{cos(x)})))`
 =  = (B \cdot cos(ln(x^{cos(x)})))$\end{center}
Таким образом

\begin{center}
 ($1)`
 =  = 0$\end{center}
//TODO: Лёша, придумай переход. У меня идеи закончились

\begin{center}
 ($5)`
 =  = 0$\end{center}
Ну вот как этот матан тебе в жизни пригодится?

\begin{center}
 ($x)`
 =  = 1$\end{center}
Господи, да для кого я вообще стараюсь

\begin{center}
 ($3)`
 =  = 0$\end{center}
Говорят

\begin{center}
 ($x^{3})`
 =  = (x^{3} \cdot ((\frac{3}{x} \cdot 1) + (0 \cdot ln(x))))$\end{center}
Не трудно заметить

\begin{center}
A = $(x^{3} \cdot ((\frac{3}{x} \cdot 1) + (0 \cdot ln(x))))$\end{center}
\begin{center}
 ($(5 \cdot x^{3}))`
 =  = ((0 \cdot x^{3}) + (5 \cdot A))$\end{center}
Любовь - это верить в его выводы без доказательств...

\begin{center}
A = $(x^{3} \cdot ((\frac{3}{x} \cdot 1) + (0 \cdot ln(x))))$\end{center}
\begin{center}
 ($(1 - (5 \cdot x^{3})))`
 =  = (0 - ((0 \cdot x^{3}) + (5 \cdot A)))$\end{center}
Руководствуясь базовой логикой, получаем

\begin{center}
A = $(x^{3} \cdot ((\frac{3}{x} \cdot 1) + (0 \cdot ln(x))))$\end{center}
\begin{center}
B = $(0 - ((0 \cdot x^{3}) + (5 \cdot A)))$\end{center}
\begin{center}
 ($ln((1 - (5 \cdot x^{3}))))`
 =  = (\frac{1}{(1 - (5 \cdot x^{3}))} \cdot B)$\end{center}
Ну вот как этот матан тебе в жизни пригодится?

\begin{center}
A = $((\frac{cos(x)}{x} \cdot 1) + ((1 \cdot (0 - sin(x))) \cdot ln(x)))$\end{center}
\begin{center}
B = $(\frac{1}{x^{cos(x)}} \cdot (x^{cos(x)} \cdot A))$\end{center}
\begin{center}
C = $((B \cdot cos(ln(x^{cos(x)}))) \cdot ln((1 - (5 \cdot x^{3}))))$\end{center}
\begin{center}
D = $(x^{3} \cdot ((\frac{3}{x} \cdot 1) + (0 \cdot ln(x))))$\end{center}
\begin{center}
E = $(0 - ((0 \cdot x^{3}) + (5 \cdot D)))$\end{center}
\begin{center}
F = $(\frac{1}{(1 - (5 \cdot x^{3}))} \cdot E)$\end{center}
\begin{center}
G = $(ln((1 - (5 \cdot x^{3}))) \cdot ln((1 - (5 \cdot x^{3}))))$\end{center}
\begin{center}
 ($\frac{sin(ln(x^{cos(x)}))}{ln((1 - (5 \cdot x^{3})))})`
 =  = \frac{(C - (F \cdot sin(ln(x^{cos(x)}))))}{G}$\end{center}
\newpage \textbf{\LARGE Глава IV.Упрощение выражения}

Отметим, что

\begin{center}
$(\frac{cos(x)}{x} \cdot 1) = \frac{cos(x)}{x}$\end{center}
При этом

\begin{center}
$(1 \cdot (0 - sin(x))) = (0 - sin(x))$\end{center}
Очевидно, что

\begin{center}
$(0 \cdot x^{3}) = 0$\end{center}
Диффиринциал от производной не далеко падает

\begin{center}
$(\frac{3}{x} \cdot 1) = \frac{3}{x}$\end{center}
Откуда

\begin{center}
$(0 \cdot ln(x)) = 0$\end{center}
ИИИИЕЕЕЕсли

\begin{center}
$(\frac{3}{x} + 0) = \frac{3}{x}$\end{center}
С другой стороны

\begin{center}
$(0 + (5 \cdot (x^{3} \cdot \frac{3}{x}))) = (5 \cdot (x^{3} \cdot \frac{3}{x}))$\end{center}
\newpage \textbf{\LARGE Глава V. Результат}

\begin{center}
A = $(\frac{cos(x)}{x} + ((0 - sin(x)) \cdot ln(x)))$\end{center}
\begin{center}
B = $(\frac{1}{x^{cos(x)}} \cdot (x^{cos(x)} \cdot A))$\end{center}
\begin{center}
C = $((B \cdot cos(ln(x^{cos(x)}))) \cdot ln((1 - (5 \cdot x^{3}))))$\end{center}
\begin{center}
D = $(0 - (5 \cdot (x^{3} \cdot \frac{3}{x})))$\end{center}
\begin{center}
E = $(\frac{1}{(1 - (5 \cdot x^{3}))} \cdot D)$\end{center}
\begin{center}
F = $(ln((1 - (5 \cdot x^{3}))) \cdot ln((1 - (5 \cdot x^{3}))))$\end{center}
$y = $$\frac{(C - (E \cdot sin(ln(x^{cos(x)}))))}{F}$
\end{document}