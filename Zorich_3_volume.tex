\documentclass[12pt,a4paper,fleqn]{article}
\usepackage[utf8]{inputenc}
\usepackage[russian]{babel}
\usepackage{amssymb, amsmath, multicol}
\usepackage{enumitem}
\usepackage{lipsum}
\usepackage{euler}
\oddsidemargin=-15.4mm
\textwidth=190mm
\headheight=-32.4mm
\textheight=277mm
\parindent=0pt
\parskip=8pt
\pagestyle{empty}
\begin{document}
$y = $$\frac{sin(ln(x^{2}))}{ln((1 - (5 \cdot x^{3})))}$

 Продиффиринцируем эту функцию

Если посмотреть на выражение под другим углом, можно получить

($x$)` = $1$\\
Не трудно заметить

($2$)` = $0$\\
Телец в козероге, поэтому

($x^{2}$)` = $(x^{2} \cdot ((\frac{2}{x} \cdot 1) + (0 \cdot ln(x))))$\\
Ну вот как этот матан тебе в жизни пригодится?

($ln(x^{2})$)` = $(\frac{1}{x^{2}} \cdot (x^{2} \cdot ((\frac{2}{x} \cdot 1) + (0 \cdot ln(x)))))$\\
//TODO: Лёша, придумай переход. У меня идеи закончились

($sin(ln(x^{2}))$)` = $((\frac{1}{x^{2}} \cdot (x^{2} \cdot ((\frac{2}{x} \cdot 1) + (0 \cdot ln(x))))) \cdot cos(ln(x^{2})))$\\
Без комментариев

($1$)` = $0$\\
Здесь могла быть ваша реклама

($5$)` = $0$\\
Если посмотреть на выражение под другим углом, можно получить

($x$)` = $1$\\
Как будет доказано в следующем семестре

($3$)` = $0$\\
Segmentation fault (core dumped)

($x^{3}$)` = $(x^{3} \cdot ((\frac{3}{x} \cdot 1) + (0 \cdot ln(x))))$\\
И хотя клуб любителей таких формул двумя блоками ниже, мы продолжаем

($(5 \cdot x^{3})$)` = $((0 \cdot x^{3}) + (5 \cdot (x^{3} \cdot ((\frac{3}{x} \cdot 1) + (0 \cdot ln(x))))))$\\
И хотя клуб любителей таких формул двумя блоками ниже, мы продолжаем

($(1 - (5 \cdot x^{3}))$)` = $(0 - ((0 \cdot x^{3}) + (5 \cdot (x^{3} \cdot ((\frac{3}{x} \cdot 1) + (0 \cdot ln(x)))))))$\\
Положим

($ln((1 - (5 \cdot x^{3})))$)` = $(\frac{1}{(1 - (5 \cdot x^{3}))} \cdot (0 - ((0 \cdot x^{3}) + (5 \cdot (x^{3} \cdot ((\frac{3}{x} \cdot 1) + (0 \cdot ln(x))))))))$\\
От коробки до нк все знают, что

($\frac{sin(ln(x^{2}))}{ln((1 - (5 \cdot x^{3})))}$)` = $\frac{((((\frac{1}{x^{2}} \cdot (x^{2} \cdot ((\frac{2}{x} \cdot 1) + (0 \cdot ln(x))))) \cdot cos(ln(x^{2}))) \cdot ln((1 - (5 \cdot x^{3})))) - ((\frac{1}{(1 - (5 \cdot x^{3}))} \cdot (0 - ((0 \cdot x^{3}) + (5 \cdot (x^{3} \cdot ((\frac{3}{x} \cdot 1) + (0 \cdot ln(x)))))))) \cdot sin(ln(x^{2}))))}{(ln((1 - (5 \cdot x^{3}))) \cdot ln((1 - (5 \cdot x^{3}))))}$\\


Упростим получившееся выражение

$(\frac{2}{x} \cdot 1)$ = $\frac{2}{x}$\\
$(0 \cdot ln(x))$ = $ln(x)$\\
$(0 \cdot x^{3})$ = $x^{3}$\\
$(\frac{3}{x} \cdot 1)$ = $\frac{3}{x}$\\
$(0 \cdot ln(x))$ = $ln(x)$\\


 Получаем выражение

$\frac{((((\frac{1}{x^{2}} \cdot (x^{2} \cdot (\frac{2}{x} + ln(x)))) \cdot cos(ln(x^{2}))) \cdot ln((1 - (5 \cdot x^{3})))) - ((\frac{1}{(1 - (5 \cdot x^{3}))} \cdot (0 - (x^{3} + (5 \cdot (x^{3} \cdot (\frac{3}{x} + ln(x))))))) \cdot sin(ln(x^{2}))))}{(ln((1 - (5 \cdot x^{3}))) \cdot ln((1 - (5 \cdot x^{3}))))}$
\end{document}