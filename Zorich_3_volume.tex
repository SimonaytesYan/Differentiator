\documentclass[12pt,a4paper,fleqn]{article}
\usepackage[utf8]{inputenc}
\usepackage[russian]{babel}
\usepackage{amssymb, amsmath, multicol}
\usepackage{enumitem}
\usepackage{lipsum}
\usepackage{euler}
\oddsidemargin=-15.4mm
\textwidth=190mm
\headheight=-32.4mm
\textheight=277mm
\parindent=0pt
\parskip=8pt
\pagestyle{empty}
\begin{document}
$y = $$(x \cdot x)$\newpage \textbf{\Huge Продиффиринцируем эту функцию}

Отметим, что

\begin{center}
($x$)`
 = $1$\end{center}
Дураку понятно, что

\begin{center}
($x$)`
 = $1$\end{center}
Не трудно заметить

\begin{center}
($(x \cdot x)$)`
 = $((1 \cdot x) + (x \cdot 1))$\end{center}
\newpage \textbf{\LARGE Упростим получившееся выражение}

\begin{center}
$(1 \cdot x)$=$x$\end{center}
\begin{center}
$(x \cdot 1)$=$x$\end{center}
\newpage \textbf{\Huge Получаем выражение}

$(x + x)$
\end{document}