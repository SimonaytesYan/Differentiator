\documentclass[12pt,a4paper,fleqn]{article}
\usepackage[utf8]{inputenc}
\usepackage[russian]{babel}
\usepackage{amssymb, amsmath, multicol}
\usepackage{enumitem}
\usepackage{lipsum}
\usepackage{euler}
\oddsidemargin=-15.4mm
\textwidth=190mm
\headheight=-32.4mm
\textheight=277mm
\parindent=0pt
\parskip=8pt
\pagestyle{empty}
\begin{document}
$y = $$\frac{x^{2}}{sin(x)}$

 Продиффиринцируем эту функцию

Вы не шокированы?

($x$)` = $1$\\
Таким образом

($2$)` = $0$\\
Кроме того

($x^{2}$)` = $(x^{2} \cdot ((\frac{2}{x} \cdot 1) + (0 \cdot ln(x))))$\\
Откуда

($x$)` = $1$\\
Дураку понятно, что

($sin(x)$)` = $(1 \cdot cos(x))$\\
Руководствуясь базовой логикой, получаем

($\frac{x^{2}}{sin(x)}$)` = $\frac{(((x^{2} \cdot ((\frac{2}{x} \cdot 1) + (0 \cdot ln(x)))) \cdot sin(x)) - ((1 \cdot cos(x)) \cdot x^{2}))}{(sin(x) \cdot sin(x))}$\\


Упростим получившееся выражение

$(\frac{2}{x} \cdot 1)$ = $\frac{2}{x}$\\
$(0 \cdot ln(x))$ = $ln(x)$\\
$(1 \cdot cos(x))$ = $cos(x)$\\


 Получаем выражение

$\frac{(((x^{2} \cdot (\frac{2}{x} + ln(x))) \cdot sin(x)) - (cos(x) \cdot x^{2}))}{(sin(x) \cdot sin(x))}$
\end{document}