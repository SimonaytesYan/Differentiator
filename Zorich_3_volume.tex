\documentclass[12pt,a4paper,fleqn]{article}
\usepackage[utf8]{inputenc}
\usepackage[russian]{babel}
\usepackage{amssymb, amsmath, multicol}
\usepackage{enumitem}
\usepackage{lipsum}
\usepackage{euler}
\oddsidemargin=-15.4mm
\textwidth=190mm
\headheight=-32.4mm
\textheight=277mm
\parindent=0pt
\parskip=8pt
\pagestyle{empty}
\begin{document}
\begin{center}
\textbf{\LARGE Исследовательская работа по теме:

Исследование функции дифференциальными методами}\end{center}\newpage\textbf{\LARGE Глава I. Функция}

\begin{center}
$y = $$\frac{sin(ln(x^{cos(x)}))}{ln((1 - (5 \cdot x^{3})))}$\end{center}
\newpage \textbf{\LARGE Глава II. Зрительный анализ функции}

Кроме того

\begin{center}
$y = $$\frac{sin(ln(x^{cos(x)}))}{ln((1 - (5 \cdot x^{3})))}$\end{center}
\newpage \textbf{\LAGRE Глава III. Дифференцирование}

Нам не объяснили на семинаре как это делать, поэтому примем на веру

\begin{center}
($x$)`
 = $1$\end{center}
Я придумал поистине удивительное доказательство этого факта, но поля этой книги слишком малы\ldots

\begin{center}
($x$)`
 = $1$\end{center}
Не так страшна производная, как её находят

\begin{center}
($cos(x)$)`
 = $(1 \cdot (0 - sin(x)))$\end{center}
//TODO: Лёша, придумай переход. У меня идеи закончились

\begin{center}
($x^{cos(x)}$)`
 = $(x^{cos(x)} \cdot ((\frac{cos(x)}{x} \cdot 1) + ((1 \cdot (0 - sin(x))) \cdot ln(x))))$\end{center}
Откуда

\begin{center}
($ln(x^{cos(x)})$)`
 = $(\frac{1}{x^{cos(x)}} \cdot (x^{cos(x)} \cdot ((\frac{cos(x)}{x} \cdot 1) + ((1 \cdot (0 - sin(x))) \cdot ln(x)))))$\end{center}
А теперь уберите детей от экранов

\begin{center}
($sin(ln(x^{cos(x)}))$)`
 = $((\frac{1}{x^{cos(x)}} \cdot (x^{cos(x)} \cdot ((\frac{cos(x)}{x} \cdot 1) + ((1 \cdot (0 - sin(x))) \cdot ln(x))))) \cdot cos(ln(x^{cos(x)})))$\end{center}
Если посмотреть на выражение под другим углом, можно получить

\begin{center}
($1$)`
 = $0$\end{center}
Телец в козероге, поэтому

\begin{center}
($5$)`
 = $0$\end{center}
С другой стороны

\begin{center}
($x$)`
 = $1$\end{center}
Спешка нужна только в армии и при ловле блох. Но не как уж ни при вычислении производной

\begin{center}
($3$)`
 = $0$\end{center}
Диффиринциал от производной не далеко падает

\begin{center}
($x^{3}$)`
 = $(x^{3} \cdot ((\frac{3}{x} \cdot 1) + (0 \cdot ln(x))))$\end{center}
Очевидно, что

\begin{center}
($(5 \cdot x^{3})$)`
 = $((0 \cdot x^{3}) + (5 \cdot (x^{3} \cdot ((\frac{3}{x} \cdot 1) + (0 \cdot ln(x))))))$\end{center}
Обоснование этого перехода было забанено редактурой

\begin{center}
($(1 - (5 \cdot x^{3}))$)`
 = $(0 - ((0 \cdot x^{3}) + (5 \cdot (x^{3} \cdot ((\frac{3}{x} \cdot 1) + (0 \cdot ln(x)))))))$\end{center}
В ближайшее время ожидаются осадки из ваших слёз от попыток понять этот переход

\begin{center}
($ln((1 - (5 \cdot x^{3})))$)`
 = $(\frac{1}{(1 - (5 \cdot x^{3}))} \cdot (0 - ((0 \cdot x^{3}) + (5 \cdot (x^{3} \cdot ((\frac{3}{x} \cdot 1) + (0 \cdot ln(x))))))))$\end{center}
Как будет доказано в следующем семестре

\begin{center}
($\frac{sin(ln(x^{cos(x)}))}{ln((1 - (5 \cdot x^{3})))}$)`
 = $\frac{((((\frac{1}{x^{cos(x)}} \cdot (x^{cos(x)} \cdot ((\frac{cos(x)}{x} \cdot 1) + ((1 \cdot (0 - sin(x))) \cdot ln(x))))) \cdot cos(ln(x^{cos(x)}))) \cdot ln((1 - (5 \cdot x^{3})))) - ((\frac{1}{(1 - (5 \cdot x^{3}))} \cdot (0 - ((0 \cdot x^{3}) + (5 \cdot (x^{3} \cdot ((\frac{3}{x} \cdot 1) + (0 \cdot ln(x)))))))) \cdot sin(ln(x^{cos(x)}))))}{(ln((1 - (5 \cdot x^{3}))) \cdot ln((1 - (5 \cdot x^{3}))))}$\end{center}
\newpage \textbf{\LARGE Глава IV.Упрощение выражения}

Положим

\begin{center}
$(\frac{cos(x)}{x} \cdot 1)$=$\frac{cos(x)}{x}$\end{center}
Руководствуясь сборником <<Задачи для подготовки к поступлению в советские ясли>>

\begin{center}
$(1 \cdot (0 - sin(x)))$=$(0 - sin(x))$\end{center}
Обоснование этого перехода было забанено редактурой

\begin{center}
$(0 \cdot x^{3})$=$0$\end{center}
Производная дураков любит

\begin{center}
$(\frac{3}{x} \cdot 1)$=$\frac{3}{x}$\end{center}
Спешка нужна только в армии и при ловле блох. Но не как уж ни при вычислении производной

\begin{center}
$(0 \cdot ln(x))$=$0$\end{center}
А теперь уберите детей от экранов

\begin{center}
$(\frac{3}{x} + 0)$=$\frac{3}{x}$\end{center}
Британские учёные доказали, что для поддержания мозга в тонусе необходимо ежедневно дифференцировать. Продолжим наше приобщение к здоровому образу жизниn

\begin{center}
$(0 + (5 \cdot (x^{3} \cdot \frac{3}{x})))$=$(5 \cdot (x^{3} \cdot \frac{3}{x}))$\end{center}
\newpage \textbf{\LARGE Глава V. Результат}

$\frac{((((\frac{1}{x^{cos(x)}} \cdot (x^{cos(x)} \cdot (\frac{cos(x)}{x} + ((0 - sin(x)) \cdot ln(x))))) \cdot cos(ln(x^{cos(x)}))) \cdot ln((1 - (5 \cdot x^{3})))) - ((\frac{1}{(1 - (5 \cdot x^{3}))} \cdot (0 - (5 \cdot (x^{3} \cdot \frac{3}{x})))) \cdot sin(ln(x^{cos(x)}))))}{(ln((1 - (5 \cdot x^{3}))) \cdot ln((1 - (5 \cdot x^{3}))))}$
\end{document}