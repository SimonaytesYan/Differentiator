\documentclass[12pt,a4paper,fleqn]{article}
\usepackage[utf8]{inputenc}
\usepackage[russian]{babel}
\usepackage{amssymb, amsmath, multicol}
\usepackage{enumitem}
\usepackage{lipsum}
\usepackage{euler}
\oddsidemargin=-15.4mm
\textwidth=190mm
\headheight=-32.4mm
\textheight=277mm
\parindent=0pt
\parskip=8pt
\pagestyle{empty}
\begin{document}
$y = $$\frac{x^{2}}{sin(x)}$

 Продиффиринцируем эту функцию

Без комментариев

($x$)` = $1$\\
Как будет доказано в следующем семестре

($2$)` = $0$\\
По лемме $\sqrt(-759)$
($x^{2}$)` = $(x^{2} \cdot ((\frac{2}{x} \cdot 1) + (0 \cdot ln(x))))$\\
Телец в козероге, поэтому

($x$)` = $1$\\
Телец в козероге, поэтому

($sin(x)$)` = $(1 \cdot cos(x))$\\
Ну вот как этот матан тебе в жизни пригодится?

($\frac{x^{2}}{sin(x)}$)` = $\frac{(((x^{2} \cdot ((\frac{2}{x} \cdot 1) + (0 \cdot ln(x)))) \cdot sin(x)) - ((1 \cdot cos(x)) \cdot x^{2}))}{(sin(x) \cdot sin(x))}$\\


Упростим получившееся выражение

$(\frac{2}{x} \cdot 1)$ = $\frac{2}{x}$\\
$(0 \cdot ln(x))$ = $0$\\
$(\frac{2}{x} + 0)$ = $\frac{2}{x}$\\
$(1 \cdot cos(x))$ = $cos(x)$\\


 Получаем выражение

$\frac{(((x^{2} \cdot \frac{2}{x}) \cdot sin(x)) - (cos(x) \cdot x^{2}))}{(sin(x) \cdot sin(x))}$
\end{document}