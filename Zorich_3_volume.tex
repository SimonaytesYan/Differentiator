\documentclass[12pt,a4paper,fleqn]{article}
\usepackage[utf8]{inputenc}
\usepackage[russian]{babel}
\usepackage{amssymb, amsmath, multicol}
\usepackage{enumitem}
\usepackage{lipsum}
\usepackage{euler}
\oddsidemargin=-15.4mm
\textwidth=190mm
\headheight=-32.4mm
\textheight=277mm
\parindent=0pt
\parskip=8pt
\pagestyle{empty}
\begin{document}
$y = $$x^{x}$

 Продиффиринцируем эту функцию

Обоснование этого перехода было забанено редактурой

($x$)` = $1$\\
Имеем

($x$)` = $1$\\
От коробки до нк все знают, что

($x^{x}$)` = $(x^{x} \cdot ((\frac{x}{x} \cdot 1) + (1 \cdot ln(x))))$\\


Упростим получившееся выражение



 Получаем выражение

$(x^{x} \cdot ((\frac{x}{x} \cdot 1) + (1 \cdot ln(x))))$
\end{document}